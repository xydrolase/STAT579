\documentclass{article}
\usepackage[margin=1in]{geometry}
\usepackage{amsmath, amssymb}
\usepackage{enumerate}
\usepackage{fancyhdr}
\usepackage{titlesec}
\usepackage{verbatim}
\usepackage{/usr/share/R/share/texmf/tex/latex/Sweave}


\pagestyle{fancy}
\fancyhead[L]{STAT 579 HW3}
\fancyhead[R]{Xin Yin}
\titleformat{\section}{\em\Large} {$\dagger$ \thesection .}{10pt}{}

\begin{document}
    \section{}
    \begin{enumerate}[(a)]
    \item
    Delve into the dataset by reading it into R, specifying the delimiter to be the Tab. 
\begin{Schunk}
\begin{Sinput}
> bill <- read.table("http://maitra.public.iastate.edu/stat579/datasets/senate-109.txt", 
+     sep = "\t", header = T)
\end{Sinput}
\end{Schunk}
    \item 

    \begin{enumerate}[i.]
    \item
    Rename the verbose name to \verb=bill_type=, and use regular expression to truncate the full bill name to its type only. 
\begin{Schunk}
\begin{Sinput}
> names(bill)[1] <- "bill_type"
> bill[, 1] <- sub("^([^_]+)_.*", "\\1", bill[, 1])
\end{Sinput}
\end{Schunk}
    \item
    And tabulate the numbers of each bill type,
\begin{Schunk}
\begin{Sinput}
> bill.tbl <- table(bill[, 1])
> print(head(bill.tbl))
\end{Sinput}
\begin{Soutput}
           Abortion Issues         Agriculture Issues 
                        17                          6 
            Appropriations        Arts and Humanities 
                        93                          1 
Budget, Spending and Taxes     Business and Consumers 
                        69                          7 
\end{Soutput}
\end{Schunk}
    and to visualize it,
\begin{Schunk}
\begin{Sinput}
> par(mar = c(12, 4.1, 4.1, 2.1))
> plot(1:length(bill.tbl), bill.tbl, xaxt = "n", typ = "h", lwd = 2, 
+     xlab = "", ylab = "Count of bill")
> axis(1, at = 1:length(bill.tbl), las = 2, labels = names(bill.tbl), 
+     cex.axis = 0.7)
> axis(2)
\end{Sinput}
\end{Schunk}
\includegraphics{hw3-004}
    \end{enumerate}
    \item One good property of how this data was arranged is that, no matter you vote against or vote with a bill, you multiply your choice by itself always yields a 1. 
    Therefore, to do the matrix multiplication, $\mathbf{XX}'$
    will give us a matrix, whose diagnol terms $\mathbf{X}_{ii}$ are the sums of votes for each bill.

    To check if there's any discrepancy in the \verb=missing_votes= column, we examine this column vector with a vector of transformed diagonal terms $100 - \text{diag}(\mathbf{XX}')$.

\begin{Schunk}
\begin{Sinput}
> vote <- as.matrix(bill[, 3:102])
> vote.count <- diag(vote %*% t(vote))
> all(100 - vote.count == bill$missing_votes)
\end{Sinput}
\begin{Soutput}
[1] TRUE
\end{Soutput}
\end{Schunk}
    Seems the data quality is pretty good and we can move on.

    \item To transform the matrix into one with relative votes with regard to the majority leaader, we first need to throw away those bills Senate Bill Frist didn't vote for.
\begin{Schunk}
\begin{Sinput}
> vote.senate <- subset(vote, vote[, 100] != 0)
> dim(vote.senate)[1] == sum(abs(vote[, 100]))
\end{Sinput}
\begin{Soutput}
[1] TRUE
\end{Soutput}
\end{Schunk}
    We extract Senator Frist's vote as a vector $v_1$. We can leverage the fact that if we take a column vector $v_2$ from matrix \verb=vote.senate= iteratively, and perform arithmatic multiplication between the two vectors $v_1, v_2$, we can get a new vector $v_1'$ whose elements represent exactly the relative vote for the corresponding Senator with regard to Bill Frist's.
\begin{Schunk}
\begin{Sinput}
> vote.leader <- vote.senate[, 100]
> vote.relative <- vote.leader * vote.senate
\end{Sinput}
\end{Schunk}
    Let's do some little experiment to check if this operation returns what we
    expected. We can compare the President's votes back to that time with Seator Bill Frist's,
\begin{Schunk}
\begin{Sinput}
> vote.cmp <- cbind(vote.senate[1:5, 2], vote.senate[1:5, 100], 
+     vote.relative[1:5, 2])
> colnames(vote.cmp) <- colnames(vote.senate)[c(2, 100, 2)]
> vote.cmp
\end{Sinput}
\begin{Soutput}
     Barack.H..Obama..IL. William.H..Bill.Frist..TN. Barack.H..Obama..IL.
[1,]                    1                         -1                   -1
[2,]                   -1                          1                   -1
[3,]                   -1                          1                   -1
[4,]                    1                         -1                   -1
[5,]                    1                         -1                   -1
\end{Soutput}
\end{Schunk}
    Hmmm... Bipartisan politicians and their games. Well, at least we've done our job right and we're happy to move on.

    \item Now we are interested in for each bill, how many Senators voted with or against, and how many novoters are there at Capitol Hill. To do this, we use \verb=outer()= function to map the relative vote matrix into a 3-dimensional array. The values on the 3rd dimension is simply the set $\{-1, 0, 1\}$. So we will have three planes expanded on 1st and 2nd dimensions, each of them is a $438 \times 100$ matrix, whose values are logical \verb=TRUE='s and \verb=FALSE='s, indicating if a senator voted against/indifferently/with a bill or not.
\begin{Schunk}
\begin{Sinput}
> vote.choice <- outer(vote.relative, -1:1, "==")
\end{Sinput}
\end{Schunk}
    Since we only care about how many people did what choices for a certain bill but not so much about who, we sum over the \verb=TRUE='s on 1st and 3rd dimension, which will end up with a $483 \times 3$ matrix. Each column of this matrix is the sum of votes for a certain bill under a particular choice (vote against, indifferently or vote with).
\begin{Schunk}
\begin{Sinput}
> vote.aggregated <- apply(vote.choice, MARGIN = c(1, 3), FUN = sum)
\end{Sinput}
\end{Schunk}

    Well, so far, this matrix still doesn't make sense to us, because from the beginning, we have chopped away the bill names and IDs. We need to aggregate our counts again, grouping by the bill types and vote choices. To do this, we use the \verb=aggregate()= function to construct a \verb=data.frame= that contains the matched count of votes, bill types as well as vote choices.
    
\begin{Schunk}
\begin{Sinput}
> vote.tbl <- aggregate(as.vector(vote.aggregated), by = list(bill_type = rep(factor(bill[vote[, 
+     100] != 0, 1]), times = 3), choice = rep(factor(c("against", 
+     "indifferently", "with")), each = dim(vote.aggregated)[1])), 
+     FUN = sum)
> names(vote.tbl)[3] <- "count"
> print(head(vote.tbl))
\end{Sinput}
\begin{Soutput}
                   bill_type  choice count
1            Abortion Issues against   520
2         Agriculture Issues against   131
3             Appropriations against  1436
4        Arts and Humanities against     7
5 Budget, Spending and Taxes against  2219
6     Business and Consumers against   290
\end{Soutput}
\end{Schunk}

    Ah, verbose numbers. So we again visualize this tabulated data, using a bar chart.
\begin{Schunk}
\begin{Sinput}
> library(ggplot2)
> p <- ggplot(vote.tbl, aes(x = bill_type, y = count, fill = choice))
> print(p + opts(axis.text.x = theme_text(angle = -90, hjust = 0)) + 
+     xlab("Bill Type") + ylab("Count of votes") + geom_bar())
\end{Sinput}
\end{Schunk}
\includegraphics{hw3-012}

    Well, looking at this plot prompts many interesting thoughts.
    For instance, budget bills are, unsurprisingly, a view into the bipartisan nature of Capitol Hill.
    And also the immigration, health issue bills etc etc.
    \end{enumerate}

    
\section{}
\begin{enumerate}[(a)]
    \item
    Read the datafile as a matrix,
\begin{Schunk}
\begin{Sinput}
> d <- as.matrix(read.table("http://www.public.iastate.edu/~maitra/stat579/datasets/fbp-img.dat", 
+     header = F))
\end{Sinput}
\end{Schunk}
    \item So we can now plot this matrix using the \verb=image()= function with 256 gray-scales.
\begin{Schunk}
\begin{Sinput}
> image(d, col = gray((1:256)/256))
\end{Sinput}
\end{Schunk}
\includegraphics{prob2-002}

    So, looks like a brain PET image to me. I'm not pretty sure if the dark region is the lesion, but anyway, we have set up the standard image for following comparisons.
    \item
    We first try to implement some utility functions that make life easier, hopefully.
\begin{Schunk}
\begin{Sinput}
> midpoints <- function(x) {
+     return(apply(matrix(x[c(1, rep(2:(length(x) - 1), each = 2), 
+         length(x))], ncol = 2, byrow = T), MARGIN = 1, FUN = mean))
+ }
> intervals <- function(x) {
+     return(matrix(x[c(1, rep(2:(length(x) - 1), each = 2), length(x))], 
+         ncol = 2, byrow = T) + matrix(rep(c(1, -1), each = length(x) - 
+         1), ncol = 2))
+ }
\end{Sinput}
\end{Schunk}
    \verb=midpoints= is the function to calculate the midpoints given $k+1$ endpoints of $k$ bins. \verb=intervals= is another simple function that helps us to build matrices whose rows are interval of bins given the endpoints. 

    Also, it is obvious that regardless of how we partition the range of values into $k$ bins, the underlying compress process is the same. So we build another function to compress a matrix into a $k$ gray-scale matrix, given the $k+1$ endpoints that can be generated in any approach.

    I was trying to implement the \verb=compress()= function that handles any constant $k$, rather than a fixed $8$. 
    To group the data points into $k$ bins, we need to find an elegent way of partitioning the values, given the intervals of the bins.
    One natural approach I can easily come up with is to put back the $k+1$ end points of $k$ bins into the vector of data points, and sort the vector. Assume we can somehow mark or tag the end points, then we can easily parition the data points into bins, given the indices of the end points in the sorted vector. 

    My little trick to mark these end points is to use complex numbers. The $k+1$ end points are tagged by adding an imaginary term, whereas normal data points only have a real component.
    Once we sort the vector, we can easily assign the pixels that fall into particular bin, by choosing values with sorted orders within certain interval, with the corresponding mid-point value. 
\begin{Schunk}
\begin{Sinput}
> compress <- function(m, endpoints) {
+     mid.pts <- midpoints(endpoints)
+     m.dim <- dim(m)
+     full.pts <- c(endpoints + (0+1i), as.vector(m))
+     pts.order <- order(full.pts)
+     marked.indices <- which(Im(full.pts[pts.order]) > 0)
+     bins <- cbind(intervals(marked.indices), mid.pts)
+     apply(bins, 1, function(x) {
+         full.pts[pts.order[seq(from = x[1], to = x[2])]] <<- x[3]
+     })
+     full.pts[pts.order[1]] <- mid.pts[1]
+     return(matrix(Re(full.pts[-(1:length(marked.indices))]), 
+         nrow = m.dim[1], ncol = m.dim[2]))
+ }
\end{Sinput}
\end{Schunk}
    The line \verb=full.pts[pts.order[1]] <- mid.pts[1]= is a small tweak that manually assigns the minimal midpoint to the minimal data point. One may refine this by constructing the \verb=bins= matrix with fixed $1$ and $N$ as the terminal indices. But since the tweak works, I'm kind of lazy to fix that.
    
    \begin{enumerate}[i.]
    \item Okay, so, now we can get our work done. First, we group the data points into equally ranged intervals (bins). Again, define a wrapper function that eases life.
\begin{Schunk}
\begin{Sinput}
> range.compress <- function(m, scale) {
+     v <- as.vector(m)
+     end.pts <- seq(min(v), max(v), length.out = scale + 1)
+     return(compress(m, end.pts))
+ }
\end{Sinput}
\end{Schunk}
    And we can compare the compressed images with $4$, $8$ and $16$ grayscales side by side, along with the original image.
\begin{Schunk}
\begin{Sinput}
> par(mar = c(2.2, 2.2, 1.2, 1.2))
> layout(matrix(c(1, 2, 3, 4), ncol = 2, byrow = 2))
> p <- sapply(c(4, 8, 16), function(x) {
+     image(range.compress(d, x), col = gray((1:256)/256), main = paste(x, 
+         "colors", sep = " "))
+ })
> image(d, col = gray((1:256)/256), main = "Original grayscale")
\end{Sinput}
\end{Schunk}
\includegraphics{prob2-006}
    \item For quantile bins, similarly, define another wrapper,
\begin{Schunk}
\begin{Sinput}
> quantile.compress <- function(m, scale) {
+     v <- as.vector(m)
+     end.pts <- quantile(v, probs = seq(from = 0, to = 1, length.out = scale + 
+         1))
+     return(compress(m, end.pts))
+ }
\end{Sinput}
\end{Schunk}
    And generate the same plots group.
\begin{Schunk}
\begin{Sinput}
> par(mar = c(2.2, 2.2, 1.2, 1.2))
> layout(matrix(c(1, 2, 3, 4), ncol = 2, byrow = 2))
> p <- sapply(c(4, 8, 16), function(x) {
+     image(quantile.compress(d, x), col = gray((1:256)/256), main = paste(x, 
+         "colors", sep = " "))
+ })
> image(d, col = gray((1:256)/256), main = "Original grayscale")
\end{Sinput}
\end{Schunk}
\includegraphics{prob2-008}
    \item
    So, what have we got? Well, obviously we can easily discern something interesting here. Two compression methods have distinct patterns of compressed images. We will have some tradeoffs between the two methods, depends on what information we are interested in.

    Some simple investigation will explain the huge difference between these two groupings, as we draw the histogram of the data points. 
\begin{Schunk}
\begin{Sinput}
> layout(c(1))
> hist(as.vector(d))
\end{Sinput}
\end{Schunk}
\begin{center}
\includegraphics[width=3in]{prob2-009}
\end{center}

    There is a remarkable peak centered at 0, possibly due to significant background noises, which we can clearly identify on the plot shown in 2(b) as those fluffy curves radiate from the brain. 
    
    For bins with equal ranges, because the peak has very thin tails, we immediately eliminate many noises by having equal bin intervals. As we can see on plot presented in 2(c).i, even with only 4 grayscales, the contour and structures of the brain can be clearly identified. 
    
    Grouping by quantiles, however, can be seriously affected by the peak. Consequently, we use most of our bins to capture the data points among the peak, whereas the true meaningful values are left with little variations. This grouping ended up with complete loss of information within the brain. 

    In fact, based on our exploration with the data, we can show that by simply removing the data points in the peak, we can greatly improve the resoution of the region in interest, in this particular case the brain.

    We arbitrarily truncated the data points below 1000 to 0 (yeah, no brainer, just a trivial example), and plot the same matrix as image.
\begin{Schunk}
\begin{Sinput}
> d.truncated <- d
> d.truncated[d.truncated < 1000] <- 0
> image(d.truncated, col = gray((1:256)/256))
\end{Sinput}
\end{Schunk}
\includegraphics{prob2-010}

    Comparing the plot above with what we got in 2(b), clearly we can see more details in the brain, which will hopefully help us to make more accurate diagnosis.
    \end{enumerate}
\end{enumerate}

\end{document}
