\documentclass[letter]{article}
\usepackage[margin=1in]{geometry}
\usepackage{amsmath, amssymb}
\usepackage{enumerate}
\usepackage{fancyhdr}
\usepackage{graphicx}
\usepackage{titlesec}

\pagestyle{fancy}
\fancyhead[L]{STAT 579 HW1}
\fancyhead[R]{Xin Yin}
\titleformat{\section}{\em\Large} {$\dagger$ \thesection .}{10pt}{}

\begin{document}
    \section{}
    \begin{verbatim}
# Read dataset
d <- read.table("http://maitra.public.iastate.edu/stat579/datasets/student-apt.dat", header=F)
names(d) <- c('disp', 'apt', 'math', 'lang', 'gknol')

# Factorize disciplines 
disp_levels <- c('tech', 'arch', 'med')
index <- 1
for (lvl in disp_levels){
    d$disp[which(d$disp == index)] <- lvl
    index <- index +1
}

d$disp <- factor(d$disp, levels=disp_levels)

# Plot pairwise scatterplot 
library(ggplot2)
plot_pairs <- matrix(data=c('apt', 'math', 'lang', 'gknol'), 
                    nrow=2, ncol=2)

for (nr in 1:2){
    filename <- paste(plot_pairs[nr, 1], '_', plot_pairs[nr, 2], '.pdf',
                sep="")

    print(filename)
    pdf(filename)

    p <- ggplot(d, aes(x=d[, plot_pairs[nr, 1]],
                       y=d[, plot_pairs[nr, 2]]))
    # output the actual graph
    print(p + geom_point(aes(colour=disp)) + 
        scale_x_continuous(plot_pairs[nr, 1]) + 
        scale_y_continuous(plot_pairs[nr, 2]))

    dev.off()
}
    \end{verbatim}
    \begin{enumerate}[(a)]
        \item Refer to code above
        \item Two scatterplots(Figure~\ref{fig:scatterplot}) can be drawn for students' quantitative
        performance in four subjects across three groups.
        \item 
        On the first subplot, a general trend can be seen that architecture
        students perform generally well both in aptitude and language.
        Technical students have larger variance among sampled individuals, but
        a vague positive correlation between their aptitude and language
        performance can be seen (I have no opinion about why so). Finally, for
        medical students, many of them clustered around left-bottom corner,
        indicating inferior performance in both aptitude and language comparing
        with two other groups. Again, a weak correlation between aptitude and
        language can be observed for medical students.

        On the second subplot, one clear indication is that majority of
        students perform mediocre in general knowledge. Their lack of general
        knowledge is independent of their mathematics performance. Another
        information can be sensed from the plot is that architecture students,
        have better averaged math performance, comparing with technical
        students, which come after, and medical students, whom on average did
        the worst performance. Generally speaking, medical students did a
        better job in general knowledge, with larger variances.
    \end{enumerate}
    \begin{figure}[htp]
        \centering
        \includegraphics[width=0.5\textwidth]{apt_lang.pdf}
        \includegraphics[width=0.5\textwidth]{math_gknol.pdf}
        \caption{Pairwise scatterplots of students' performance in 4
        subjects across 3 groups}
        \label{fig:scatterplot}
    \end{figure}

    \section{}
    \begin{verbatim}
# Read dataset
cars <- read.table("http://maitra.public.iastate.edu/stat579/datasets/cars.dat", header=T)

attach(cars)

# (why would stupid Americans keep resisting to switch to metric system?)
# 1 mph = 1.4666667 ft/sec
# 1 mile = 1.6093 km
# 1 foot = 0.3048 m
speed_ftps <- speed * 1.46667
# Plot speed against distance (I guess this is the braking distance, though no 
# such information is mentioned in the question).
pdf('speed_vs_dist_ft.pdf')
plot(x=speed_ftps, y=dist, 
    xlab='Speed (feet/s)', ylab='Braking distance (feet)',
    main='Plotting braking distance against speed')
dev.off()

# Unit conversion, again.
speed_mps <- speed * 1.6093 * 1000 / 3600 
dist_m <- dist * 0.3048

detach(cars)

pdf('speed_vs_dist_m.pdf')
plot(x=speed_mps, y=dist_m, 
    xlab='Speed (m/s)', ylab='Braking distance (meters)',
    main='Plotting braking distance against speed')
dev.off()
    \end{verbatim}
    Presented above is the R code for problem 2. 

    Somehow, something is pretty vague with this dataset. The interpretation of
    ``distance'' in this dataset is ambiguous. However, based on the
    relationship between speed and distance demonstrated in two following
    plots, we can conjecture that ``distance'' here refers to the braking
    distance of car given a particular speed. 

    \begin{figure}[htp]
        \centering
        \includegraphics[width=0.4\textwidth]{speed_vs_dist_ft.pdf}
        \includegraphics[width=0.4\textwidth]{speed_vs_dist_m.pdf}
        \caption{Vehicle stoppage distance versus vehicle speed}
        \label{fig:dist_speed}
    \end{figure}

    If we assume that the ``distance'' on $Y$-axis of both plots
    (Figure~\ref{fig:dist_speed}) above indeed
    implies braking distance or vehicle stoppage distance, one simple
    conclusion can be made, based on the plots, that there is a high
    correlation between vehicle speed and its braking distance. It is not easy
    to conclude, merely given these two plots, that if this relationship is
    linear because of the variance observed on the plots. But in general, the
    faster one vehicle goes, it takes longer distance to come to a complete
    stop. Unit conversion from one system to another didn't change the
    interpretation as they are proportionally scaled. 

    \section{}
    \begin{verbatim}
# R program to problem 3 of HW1 / STAT 579

# Convert temperature from Celsius scale to Fahrenheit
temp.f <- pressure$temperature * 9 / 5 + 32
pres.f <- data.frame(temperature=temp.f, pressure=pressure$pressure)

# Linear regression
fit <- lm(temperature ~ 0 + pressure, data=pres.f)
print(summary(fit))

# Plot temperature against pressure with fitted line
pdf('temp_pres.pdf')
plot(x=pres.f$pressure, y=pres.f$temperature)
# Draw regression line
abline(a=0, b=fit$coefficients[1])
dev.off()

pdf('fitted_residual.pdf')
plot(x=fit$fitted.values, y=fit$residuals,
    xlab='Fitted values', ylab='Residuals',
    main='Residual plot of linear fit of temperature against pressure')
dev.off()

# A new model...
pres.full <- data.frame(temperature=temp.f,
    pres=pressure$pressure,
    pres.sq=pressure$pressure^2,
    pres.cubic=pressure$pressure^3)

# Fit it with a MLR
fit.full <- lm(temperature ~ pres + pres.sq + pres.cubic, data=pres.full)
pdf('pres_mlr.pdf')
plot(x=pres.f$pressure, y=pres.f$temperature, 
    xlab='Pressure (mm/Hg)', ylab='Temperature (F)',
    main='Fit water vapor temperature against pressures')
lines(x=pres.full$pres, y=fit.full$fitted.values)
dev.off()

# Residual plot for MLR
pdf('fitted_residuals_mlr.pdf')
plot(x=fit.full$fitted.values, y=fit.full$residuals,
    xlab='Fitted values', ylab='Residuals',
    main='Residual plot of multiple linear regression of temperature against pressures')
dev.off()
    \end{verbatim}
    R code to solve problem 3 is presented as above.

    It is interesting though, that \verb=help(pressure)= suggests that this
    dataset was recorded with \verb=temperature= as its independent variable.
    But, it won't matter if we do the other way around since increment in
    container pressure, given a certain amount of water vapor, will result in
    increased temperature of water vapor. 

    \begin{enumerate}
    \item[(c)]
    Plotting temperature against pressure (in Fahrenheit), we got
    Figure~\ref{fig:res_plot}.
    \begin{figure}[htp]
        \centering
        \begin{minipage}[b]{0.45\textwidth}
        \includegraphics[width=0.9\textwidth]{temp_pres.pdf}
        \caption{Water vapor temperature versus container pressure}
        \label{fig:temp_pres}
        \end{minipage}
        \begin{minipage}[b]{0.45\textwidth}
        \includegraphics[width=0.9\textwidth]{fitted_residual.pdf}
        \caption{Residual plot of residuals versus fitted temperatures}
        \label{fig:res_plot}
        \end{minipage}
    \end{figure}
    \item[(d)]
    Fit temperature versus pressure with a linear model, using 
    \verb|lm(temperature ~ 0 + pressure, data=pres.f)| without the intercept
    gave us,
    \begin{verbatim}
Call:
lm(formula = temperature ~ 0 + pressure, data = pres.f)

Residuals:
   Min     1Q Median     3Q    Max 
-300.2  122.0  247.1  345.2  394.7 

Coefficients:
         Estimate Std. Error t value Pr(>|t|)    
pressure   1.2161     0.2516   4.834 0.000133 ***
---
Signif. codes:  0 ‘***’ 0.001 ‘**’ 0.01 ‘*’ 0.05 ‘.’ 0.1 ‘ ’ 1 

Residual standard error: 275.8 on 18 degrees of freedom
Multiple R-squared: 0.5649, Adjusted R-squared: 0.5408 
F-statistic: 23.37 on 1 and 18 DF,  p-value: 0.0001330 
    \end{verbatim}
    The line of fitted values was plotted on Figure~\ref{fig:temp_pres} as the
    solid line. Adjusted $R^2 = 0.5408$ suggests that the linear relationship
    is not prominent, which can be supported by merely looking at
    Figure~\ref{fig:temp_pres}. The curve of temperature versus pressure, on
    common sense, approximately follows a log-shaped curvature. 

    In addition, because the temperature is measured on Fahrenheit scale, which
    has a $32$ degree offset over Celcius degree, the linear regression has to
    compesate for that offset. The linear relationship reported by linear
    regression suggests that pressure increase, measured by one unit of mm
    increase of mercury, results in about $1.2161$ degree (F) increase in water
    vapor temperature. 
    \item[(e)]
    If we draw a residual plot (Figure~\ref{fig:res_plot}) of the linear regression in part (c), we can
    clearly see that the simple linear regression didn't capture the
    relationship between temperature and pressure nicely. For pressure values
    smaller than 800mm of mercury, the linear regression tends to underestimate
    the temperature (which can also be seen on Figure~\ref{fig:temp_pres}
    because the fitted line is below the curvature for most of the values). And
    after that point, the simple linear equation $T = 1.2146 P$ tends to
    overestimate the temperature. Huge residuals that deviate from our fitted
    values hint that the simple linear model is way from a good and practical
    model.
    \item[(g)]
    We can fit temperature versus pressure with a polynomial model with a order
    of 3, trying to interpret the resulting water vapor temperature as linear
    combination of its pressure to the power of 3.
    
    Call \verb=summary(fit.full)=, we had,
    \begin{verbatim}
> summary(fit.full)
Call:
lm(formula = temperature ~ pres + pres.sq + pres.cubic, data = pres.full)

Residuals:
    Min      1Q  Median      3Q     Max 
-181.39  -56.54   -2.31   72.88  121.93 

Coefficients:
              Estimate Std. Error t value Pr(>|t|)    
(Intercept)  2.134e+02  2.966e+01   7.195  3.1e-06 ***
pres         3.417e+00  7.671e-01   4.454 0.000464 ***
pres.sq     -8.207e-03  2.826e-03  -2.904 0.010898 *  
pres.cubic   5.842e-06  2.473e-06   2.362 0.032094 *  
---
Signif. codes:  0 ‘***’ 0.001 ‘**’ 0.01 ‘*’ 0.05 ‘.’ 0.1 ‘ ’ 1 

Residual standard error: 98.98 on 15 degrees of freedom
Multiple R-squared: 0.8011, Adjusted R-squared: 0.7613 
F-statistic: 20.14 on 3 and 15 DF,  p-value: 1.618e-05 
    \end{verbatim}
    This suggests that the intercept term and the pressure term are more
    significant, comparing with the squared and cubed pressures. It is
    reasonable, because the intercept tries to capture the offset introduced by
    converting Celcius scale to Farenheit, and the pressure term, with a
    coefficient $\beta_1 = 3.147$ is the major term that correspond to the
    response in temerature as we vary the pressure. The squared and cubed terms
    are mostly for compensating the over-fitted values, by the pressure term, of those points reside at the plateau of observed curvature.
    \item[(h)]
    Plot the fitted values in multiple linear regression on the graph we had in
    (c) ended up with plot below (Figure~\ref{fig:pres_mlr}).
    
    Clearly, we have a better fit as the fitted values are closer to the
    observed points, but it's not perfect. Because the intercept term we
    introduced tend to way overestimate temperatures when pressure is close to
    zero. 
    \item[(g)]
    \begin{figure}[htp]
        \centering
        \begin{minipage}[b]{0.45\textwidth}
        \includegraphics[width=0.8\textwidth]{pres_mlr.pdf}
        \caption{Water vapor temperature versus pressure with fitted values of
        MLR}
        \label{fig:pres_mlr}
        \end{minipage}
        \begin{minipage}[b]{0.45\textwidth}
        \includegraphics[width=0.8\textwidth]{fitted_residuals_mlr.pdf}
        \caption{Residual plots of the MLR}
        \label{fig:res_plot_mlr}
        \end{minipage}
    \end{figure}
    Again, a residual plot(Figure~\ref{fig:res_plot_mlr}) can demonstrates how our fit deviates from the data.
    Huge residue variance can be observed at where fitted values are close to
    $200$, because of the boasted intercept term elevates fitted temperatures
    given even a small pressure. After that, the
    residual shrinks, which corresponds to a better fit when range of pressure
    getting beyond 0.
    \end{enumerate}
\end{document}
