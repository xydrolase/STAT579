\documentclass[letter]{article}
\usepackage[margin=1in]{geometry}
\usepackage{amsmath, amssymb}
\usepackage{enumerate}
\usepackage{fancyhdr}
\usepackage{graphicx}
\usepackage{titlesec}

\pagestyle{fancy}
\fancyhead[L]{STAT 579 HW1}
\fancyhead[R]{Xin Yin}
\titleformat{\section}{\em\Large} {$\dagger$ \thesection .}{10pt}{}

\begin{document}
    \section{}
    The R code for this problem is,
    \begin{verbatim}
sd.short <- function(smp){
    n <- length(smp)
    smp.mean <- sum(smp) / n
    smp.sq.mean <- sum(smp^2) / n 

    biased.sd <- smp.sq.mean - smp.mean^2
    return (sqrt(biased.sd * n / (n-1)))
}

sd.long <- function(smp){
    n <- length(smp)
    smp.mean <- sum(smp) / n 
    smp.ss <- (smp - smp.mean)^2

    smp.dev <- sqrt(sum(smp.ss)/(n-1))
    return (smp.dev)
}

num <- rep(c(1001, 1002), times=5) 
print (c(sd.short(num), sd.long(num), sd(num)))

    [1] 0.5270463 0.5270463 0.5270463

num.1 <- rep(c(100000000000001, 100000000000002), times=5)
print (c(sd.short(num.1), sd.long(num.1), sd(num.1)))

    [1] 0.0000000 0.5270463 0.5270463
    \end{verbatim}
    The output for two \verb=print= calls is pasted within the code segment.

    To solve the exactly same problem in Excel, we can call \verb=STDEV= 
    function in any cell, \emph{e.g.} \verb|=STDEV(1001, 1002, 1001...)|.
    This gives us two standard deviations,
    \begin{verbatim}
    0.527046
    0.527046
    \end{verbatim}

    The SAS code is,
    \begin{verbatim}
data smp;
	input x y;
	datalines;
	1001 100000000000001
	1002 100000000000002
	1001 100000000000001
	1002 100000000000002
	1001 100000000000001
	1002 100000000000002
	1001 100000000000001
	1002 100000000000002
	1001 100000000000001
	1002 100000000000002
run;

proc means stddev data=num;
	var x ;
run;
    \end{verbatim}
    which yields,
    \begin{verbatim}
                              The MEANS Procedure

                           Variable         Std Dev
                           ------------------------
                           x              0.5270463
                           y                      0
                           ------------------------
    \end{verbatim}
    Apparently, SAS has precision problem dealing with large numbers, while
    Excel and R are doing fine. 
    Using the short formula to calculate standard deviation also has the same
    precision problem.

    \section{}
    Source code for solving problem 2 is presented below. Output for summary measurements are pasted with correspondence to the code.

    \begin{verbatim}
# Problem 2
wind <- read.csv("wind.csv", header=T)

# Summary of measurements
mean(wind)

       Spring    Summer    Autumn    Winter 
    176.66667  80.83333 200.00000 238.33333 

sd(wind)

       Spring    Summer    Autumn    Winter 
    123.97458  72.92067  94.00193  86.63752 

apply(wind, 2, quantile, probes=c(0, 0.25, 0.5, 0.75, 1))

         Spring Summer Autumn Winter
    0%        0     10     30   50.0
    25%      55     20    155  205.0
    50%     185     35    215  255.0
    75%     275    150    260  297.5
    100%    350    190    350  340.0

apply(wind, 2, IQR)

    Spring Summer Autumn Winter 
     220.0  130.0  105.0   92.5

n <- dim(wind)[2]
for (i in 1:n){
    if (dev.cur()[1] == 1){
        pdf('wind.pdf')

        plot(x=i/2*cos(wind[,i]*pi/180), y=i/2*sin(wind[,i]*pi/180),
            xlab='Cos(wind angle) scaled by season', 
            ylab='Sin(wind angle) scaled by season',
            xlim=c(-n/2, n/2), ylim=c(-n/2, n/2),
            col=i, pch=i,
            main="Wind direction by season")
    }
    else {
        points(x=i/2*cos(wind[,i]*pi/180), y=i/2*sin(wind[,i]*pi/180),
                col=i, pch=i)
    }

    lines(x=i/2*cos(seq(0, 2*pi, length.out=180)),
            y=i/2*sin(seq(0, 2*pi, length.out=180)),
            col=i, lty='dotted'
          )
}

abline(v=c(0), col='lightgray', lty='dotted')
abline(h=c(0), col='lightgray', lty='dotted')
legend("topright", names(wind), col=1:n, pch=1:n)
dev.off()
    \end{verbatim}

    \begin{enumerate}
        \item[(b)] Please refer to the code above for summary measurements.
        \item[(c)] Though the data is in angular form, all the values are
        within $[0, 360]$ range. Therefore, some of the reported measurements are still
        meaningful statistics to understand the general trend and variance of
        wind directions across four season, \emph{e.g.} the means of wind
        direction quantify the averaged wind direction in different seasons,
        and the sample deviations measure the variation of wind directions. 

        The quartiles, if examined independently given a particular season,
        don't make too much sense. But if we compare the quartiles across four
        seasons, we can get some sense about how wind direction ranges and varies in
        different seasons.
        \item[(d)]
        The code presented above plots the wind directions as a bivariate plot. 
        I tried to plot the $\cos(\theta), \sin(\theta)$ pair at first, but
        many data points overlap with each other across different seasons. So I
        plotted the wind directions as the $\cos()$ and $\sin()$ values, scaled
        by a radius factor to distinguish different seasons.
        Note that for wind direction notations, 0 denotes north wind, and 90
        denotes east wind etc etc. So in order to visualize the wind direction
        correctly using the $\cos$ and $\sin$ functions, we have to transform
        the angles using following formula:
        \[
        \hat \theta = 360-(\theta-90)
        \]

        \begin{figure}[htp]
        \centering
        \includegraphics[width=0.5\textwidth]{wind.pdf}
        \caption{Wind directions in four seasons}
        \end{figure}

        From the innermost to outermost, four circles denote the four seasons
        from spring to winter. And you can still read the direction of the
        wind, regarding the whole plot as if a polar coordinate system. Each
        point represents the direction that wind blows from, and is colored
        based on season.

        Based on the this graph, we can clearly see that:
        \begin{enumerate}[i)]
        \item (South)east wind is rare in spring
        \item South and north winds are predominant during summer
        \item Wind varies significantly during the fall that it can come from
        any direction. Based on this graph however, diagonal winds tend to be
        more rare.
        \item During winter, wind tends to blow from the west, as few dots can
        be seen on the east.
        \end{enumerate}
    \end{enumerate}
    

    
\end{document}
